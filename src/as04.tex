\documentclass[12pt]{article}
\title{MA511 A04}

\usepackage{geometry}
\usepackage{amsmath}
\usepackage{tikz}

\geometry{hmargin={2cm,0.8in},height=8in}
\geometry{height=10in}

\usepackage{paralist}
\usepackage{enumerate}
\usepackage{amsfonts}
\usepackage{verbatim}

\pagestyle{empty}


\setlength{\parindent}{0pt}

\newcommand{\ds}{\displaystyle}
\newcommand{\ra}{\rightarrow}
\newcommand{\Ra}{\Rightarrow}
\newcommand{\la}{\leftarrow}
\newcommand{\La}{\Leftarrow}

\newtheorem{thm}{Theorem}%[section]
\newtheorem{lem}{Lemma}%[theorem]
\newtheorem{prop}{Proposition}%[theorem]
\newtheorem{cor}{Corollary}%[theorem]
\newtheorem{defn}{Definition}


\begin{document}
\begin{flushleft}
{\sc \Large Real Analysis} \\
\medskip
Assignment 04\\
May 27, 2020
\hfill Name: Feng Qiao\\

\setdefaultleftmargin{0pt}{}{}{}{}{}

\begin{defn}
(Ordering of the integers). Let \(n\) and \(m\) be integers. We say that \(n\) is greater than or equal to \(m\), and write \(n \geq m\) or \(m \leq n\),  iff we have \(n = m + a\) for some natural number \(a\). We say that \(n\) is strictly greater than \(m\), and write \(n > m\) or \(m < n\), iff \(n \geq m\) and \(n \neq m\).
\end{defn}

\begin{lem}
(Properties of order). Let \(a, b, c\) be integers.
\begin{enumerate}[(a)]
  \item \(a > b\) if and only if \(a\)---\(b\) is a positive natural number.
    \item (Addition preserves order) If \(a > b\), then \(a + c > b + c\).
    \item (Positive multiplication preserves order) If \(a > b\) and \(c\) is positive, then \(ac > bc\).
    \item (Negation reverses order) If \(a > b\), then \(-a < -b\).
    \item (Order is transitive) If \(a > b\) and \(b > c\), then \(a > c\).
    \item (Order trichotomy) Exactly one of the statements \(a > b\), \(a < b\), or \(a = b\) is true.
\end{enumerate}
\end{lem}

\textbf{Proof:}

\begin{enumerate}[(a)]
  \item If \(a-b=p\) for some positive natural number \(p\).

    First, we use the identity function to map \(p\) to an integer: \(p \equiv p\)---0.

    Because addition is wel-defined, we have \((a + (-b)) + b =  b + p\).

    By laws of algebra, we have \(a + ((-b)+b)=a + 0 =a=b+p\). Then \(a>b\).

    Next we prove \(a>b\) only if \(a-b\) is a positive natural number. Let \(a -b =n\).

    When \(n =0\), we have
    \begin{equation*}
      \begin{aligned}
        a + (-b) + b &= 0 + b \\
        a + 0 &=b\\
        a &= b
      \end{aligned}
    \end{equation*}

    When \(n = -q\) for some positive natural number \(q\), we have
    \begin{equation*}
      \begin{aligned}
        a + (-b) + b &= b+(-q) \\
        a &= b + (-q)
      \end{aligned}
    \end{equation*}
    Suppose \(n=-r\) for some positive natural number \(r\). Then we have
    \begin{equation*}
      \begin{aligned}
        b + r &= b + (-q)\\
        (-b) + b + r &= (-b) + b + (-q)\\
        r &= -q
      \end{aligned}
    \end{equation*}
    Because \(r\) and \(q\) are positive natural number, \(n=-r\) can not be true.
    By the trichotomy of integers, \(n\) can only be equal to a positive natural number.
    Therefore, \(a > b\) if and only if \(a-b\) is a positive natural number.
\end{enumerate}

\end{flushleft}

\end{document}
