\documentclass[12pt]{article}
\title{MA511 A02}

\usepackage{geometry}
\usepackage{amsmath}
\usepackage{tikz}

\geometry{hmargin={2cm,0.8in},height=8in}
\geometry{height=10in}

\usepackage{paralist}
\usepackage{enumerate}
\usepackage{amsfonts}
\usepackage{verbatim}

\pagestyle{empty}


\setlength{\parindent}{0pt}

\newcommand{\ds}{\displaystyle}
\newcommand{\ra}{\rightarrow}
\newcommand{\Ra}{\Rightarrow}
\newcommand{\la}{\leftarrow}
\newcommand{\La}{\Leftarrow}

\newtheorem{thm}{Theorem}%[section]
\newtheorem{lem}{Lemma}%[theorem]
\newtheorem{prop}{Proposition}%[theorem]
\newtheorem{cor}{Corollary}%[theorem]
\newtheorem{defn}{Definition}


\begin{document}
\begin{flushleft}
{\sc \Large Real Analysis} \\ 
\medskip
Assignment 02\\
May 21, 2020
\hfill Name: \underline{\hspace{3.0in}} \\

\setdefaultleftmargin{0pt}{}{}{}{}{}

\textit{Exercise 3.1.5} Let \(A,B\) be sets. Show that the three statements \(A\subseteq B\), \(A \cup B = B\), \(A \cap B = A\) are logically equivalent (any one of them implies the other two).\vfill 

\textbf{Proof:}
Let \(A\) and $B$ be sets. %anything between \(...\) or $...$ is math text.  There is no difference between the two commands.  Use the % to comment the rest of the line.
We want to show that if \(A\subseteq B\), then \(A\cup B=B\).  (To show two sets are equal, we must show they are subsets of each other.)

Let \(x\in A\cup B\), then \(x\in A\) or \(x\in B\).  Since \(A\subseteq B\), it must be the case that \(x\in B\) or \(x\in B\).  Therefore, \(A\cup B\subseteq B\).

Let \(x\in B\), then \(x\in B\) or \(x\in A\) so \(x\in A\cup B\).  Therefore, \(B\subseteq A\cup B\).  Thus, if \(A\subseteq B\), then \(A\cup B=B\).\vfill

Prove if \(A\cup B=B\), then \(A\cap B=A\).\vfill

Prove if \(A\cap B=A\), then \(A\subseteq B\).\vfill


\end{flushleft}

\end{document}
