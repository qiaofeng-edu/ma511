\documentclass[12pt]{article}
\title{MA511 A03}

\usepackage{geometry}
\usepackage{amsmath}
\usepackage{tikz}

\geometry{hmargin={2cm,0.8in},height=8in}
\geometry{height=10in}

\usepackage{paralist}
\usepackage{enumerate}
\usepackage{amsfonts}
\usepackage{verbatim}

\pagestyle{empty}


\setlength{\parindent}{0pt}

\newcommand{\ds}{\displaystyle}
\newcommand{\ra}{\rightarrow}
\newcommand{\Ra}{\Rightarrow}
\newcommand{\la}{\leftarrow}
\newcommand{\La}{\Leftarrow}
\newcommand{\fasb}{
f(A) \cap f(B)
}
\newcommand{\fab}{
f(A \cap B)
}
\newcommand{\fxab}{
    f(A) \setminus f(B)
}
\newcommand{\faxb}{
    f(A \setminus B)
}
\newcommand{\fxstTwo}{
\fxab \subseteq \faxb
}

\newtheorem{thm}{Theorem}%[section]
\newtheorem{lem}{Lemma}%[theorem]
\newtheorem{prop}{Proposition}%[theorem]
\newtheorem{cor}{Corollary}%[theorem]
\newtheorem{defn}{Definition}


\begin{document}
\begin{flushleft}
{\sc \Large Real Analysis} \\ 
\medskip
Assignment 03\\
May 26, 2020
\hfill Name: \underline{Feng Qiao} \\

\setdefaultleftmargin{0pt}{}{}{}{}{}

\textit{Exercise 3.4.3} Let \(A,B\) be two subsets of a set \(X\), and let \(f : X \ra Y\) be a function. Show that \(f(A \cap B) \subseteq f(A) \cap f(B)\), that \(f(A)\setminus f(B) \subseteq f(A\setminus B)\), \(f(A \cup B) = f(A) \cup f(B)\). For the first two statements, is it true that the \(\subseteq\) relation can be improved to =?

\end{flushleft}

\textbf{Proof:}
\renewcommand{\labelenumi}{\alph{enumi})}
\renewcommand{\labelenumii}{\arabic{enumii})}
\begin{enumerate}
    \item \(f(A \cap B) \subseteq f(A) \cap f(B)\)\\
    For any \(y \in f(A \cap B)\), by the definition of function, there exists some \(x \in A \cap B\) such that \(f(x) = y \).
    Then \(x \in A\) and \(x \in B\). Then we have \(f(x) \in f(A)\) and \(f(x) \in f(B)\). Thus, \(f(x) \in f(A) \cap f(B)\).
    Then \(y \in f(A) \cap f(B)\). Therefore, \(f(A \cap B) \subseteq f(A) \cap f(B)\).
    \item \(\fab = \fasb \)\\
    The statement is true if \(\fasb \subseteq \fab\). We want to prove that for any \(y \in \fasb\),
    we can get \(y \in \fab\).
    \begin{equation}
        \begin{aligned}
        y \in \fasb & \implies y \in f(A) \quad \text{and} \quad y \in f(B)\\
        & \implies y = f(x_1) \quad \text{and} \quad y = f(x_2) \quad
        (x_1 \in A \quad \text{and} \quad  x_2 \in B)
        \end{aligned}
    \end{equation}

    \begin{equation}
        \begin{aligned}
        y \in \fab & \implies y = f(x_3) \quad (x_3 \in A \cap B)\\
        & \implies y = f(x_3) \quad (x_3 \in A \quad \text{and} \quad x_3 \in B)
        \end{aligned}
    \end{equation}

    If \(A = B\), it can will always be true that \(x_1 \in A \cap B\). Therefore, \(A = B\) is a sufficient condition for \(\fab = \fasb\).

    \item \(\fxstTwo\)

    For any \(y \in \fxab\), we have \(y \in f(A)\) and \(y \notin f(B)\).

    Then there exists some \(x \in A\) such that \(f(x)=y\).

    Suppose that \(x \in B\), then \(y = f(x) \in f(B)\). It contradicts \(y \notin f(B)\).
    Thus, \(x \in A\) and \(x \notin B\); \(x \in A \setminus B\). Therefore, \(y = f(x) \in \faxb\).

    Therefore, \(\fxstTwo\)


\end{enumerate}

\end{document}

