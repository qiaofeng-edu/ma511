\documentclass[12pt]{article}
\title{MA511 A05}

\usepackage{geometry}
\usepackage{amsmath}
\usepackage{tikz}

\geometry{hmargin={2cm,0.8in},height=8in}
\geometry{height=10in}

\usepackage{paralist}
\usepackage{enumerate}
\usepackage{amsfonts}
\usepackage{verbatim}

\pagestyle{empty}


\setlength{\parindent}{0pt}

\newcommand{\ds}{\displaystyle}
\newcommand{\ra}{\rightarrow}
\newcommand{\Ra}{\Rightarrow}
\newcommand{\la}{\leftarrow}
\newcommand{\La}{\Leftarrow}

\newtheorem{thm}{Theorem}%[section]
\newtheorem{lem}{Lemma}%[theorem]
\newtheorem{prop}{Proposition}%[theorem]
\newtheorem{cor}{Corollary}%[theorem]
\newtheorem{defn}{Definition}


\begin{document}
\begin{flushleft}
{\sc \Large Real Analysis} \\ 
\medskip
Assignment 05\\
May 28, 2020
\hfill Name: \underline{Feng Qiao} \\

\setdefaultleftmargin{0pt}{}{}{}{}{}

\textit{Exercise 4.3.5} Prove that \(2^N \geq N\) for all positive integers \(N\). (Hint: use induction.)

\textbf{Proof:}

When \(N=1\), \(2^1=2^0 \times 2 = 1 \times 2 = 2\). Thus, \(2^N \ge N \)  is true when \(N=1\).

Suppose when \(N = n\), \(2^n \ge n\) where \(n\) is a positive integer.
We know \(n \ge 1\).

When \(N = n +1\), \(2^{n+1} = 2^n \times 2\). Because \(2 = 1 + 1\), by laws of algebra,
\begin{equation*}
  \begin{aligned}
    2^{n+1} & = 2^n \times (1 + 1)\\
    &= (2^n \times 1) + (2^n \times 1)\\
    &= 2^n  + 2^n\\
  \end{aligned}
\end{equation*}
Because addition of intergers preserves order, we have \(2^{n+1} \ge 2^n + n \ge n + n \ge n + 1\).
Thus \(2^N \ge N\) when \(N = n + 1\).

Therefore, \(2^N \geq N\) for all positive integers \(N\).


\end{flushleft}

\end{document}
